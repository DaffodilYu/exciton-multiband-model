\documentclass[
aps,
prl,
groupedaddress,
superscriptaddress,
floatfix,
%showpacs,
%showkeys,
%draft,
%preprint,
%reprint,
%twocolumn,
notitlepage
]{revtex4-1}

% documentclass options explained:
% 'draft': mark overfull boxes with black boxes
% 'showpacs': PACS codes
% 'showkeys': keywords
% 'floatfix'  table and figure placement

\usepackage{graphicx}% Include figure files
%\usepackage[dvips]{graphicx}
\usepackage{amsmath,amssymb}
\usepackage{braket}
%\usepackage{pscyr}
%\usepackage[utf8]{inputenc} % needed for Cyrillic fonts
%\usepackage[T2A]{fontenc} % needed for Cyrillic fonts
\usepackage{hyperref} % hypertext capabilities
\usepackage{subfigure}
%\usepackage{dcolumn} % Align table columns on decimal point
\usepackage{bm} % bold math
\usepackage{mathtools} % dcases (uncompressed fractions)
\usepackage{float}

% My definitions
\def\sech{{\rm sech}}
\def\rot{{\rm rot}}
\def\div{{\rm div}}
\def\arcsinh{{\rm arcsinh}}
\def\Re{{\rm Re\,}}
\def\Im{{\rm Im\,}}
\def\arccot{{\rm arccot}}
\def\p{\partial}
\DeclareMathOperator{\sgn}{sgn}

%%% Control of figures: comment line \figurestrue for text-only
\newif\iffigures
%\figurestrue

\iffigures
\usepackage{cprotect}
\fi

\begin{document}
	
%\preprint{}

%\title{Integrals}
\title{Excitons in MoS2 from the multiband model}

\author{D. R. Gulevich} 
\affiliation{ITMO University, St. Petersburg 197101, Russia}

\date{\today}

\begin{abstract} 
In collaboration with Y. Zhumagulov, V. Zalipaev, A. Vagov and V. Perebeinos.
\end{abstract}

%\maketitle
{\let\newpage\relax\maketitle}

%%%%%%%%%%%%%%55
%\iffalse
%%%%%%%%%%%%%%%5
%--------------------------------------------------------------------------------------------
%--------------------------------------------------------------------------------------------
\section{Introduction}
%--------------------------------------------------------------------------------------------%--------------------------------------------------------------------------------------------

Here we will calculate exciton states using the model~\cite{Peeters-PRB-2017}. 
See the Jupyter notebook \verb|multiband.ipynb|.

%-----------------------------------------------------------------------------------
%-----------------------------------------------------------------------------------
\section{Model}
%-----------------------------------------------------------------------------------
%-----------------------------------------------------------------------------------

Consider only the excitons with zero center-of-mass momentum $\pmb{K}=\pmb{k}_e+\pmb{k}_h=0$. Introducing $\pmb{k}\equiv \pmb{k}_e = -\pmb{k}_h$,
and using transformation to the polar coordinates
$$
\p_x = \cos\varphi\; \p_r - \frac{\sin\varphi}{r}\,\p_\varphi, \quad
\p_y = \sin\varphi\; \p_r + \frac{\cos\varphi}{r}\,\p_\varphi,
$$
exciton Hamiltonian (5) of Ref.~\cite{Peeters-PRB-2017} can be written in the following operator form
\setlength\arraycolsep{7pt}
\renewcommand*{\arraystretch}{2}
\begin{equation}
\hat{H} =
\begin{pmatrix}
-V(r) &  e^{i\tau_h \varphi}D^-_h & -e^{-i\tau_e \varphi}D^+_e & 0 \\
e^{-i\tau_h\varphi}D^+_h & \Delta_h-V(r) & 0 & -e^{-i\tau_e\varphi}D^+_e \\
-e^{i\tau_e\varphi}D^-_e & 0 & -\Delta_e-V(r) & e^{i\tau_h\varphi}D^-_h \\
0 & -e^{i\tau_e\varphi}D^-_e & e^{-i\tau_h\varphi}D^+_h & \Lambda-V(r)
\end{pmatrix},
\label{H}
\end{equation}
where $V(r)$ is the Keldysh potential 
\begin{equation}
V(r) = \mathcal{V}_0\,\frac{\pi}{2 r_0} \left[ H_0\left(\frac{r}{r_0}\right)-Y_0\left(\frac{r}{r_0}\right) \right]  = \int_0^{\infty}
\tilde{V}(k)J_0(k r) k dk,\quad 
\tilde{V}(k)=\frac{\mathcal{V}_0}{k(1+r_0 k)}
\label{Keldysh}
\end{equation}
and we have defined differential operators
$D^\pm_{e,h} \equiv i\tau_{e,h}\p_r \pm \p_\varphi/r$
and quantities
$\Delta_{e,h} \equiv \Delta -\lambda s_{e,h}\tau_{e,h}$,
$\Lambda \equiv \lambda(s_e \tau_e-s_h\tau_h)$.
Here we have expressed quantities with dimension of energy ($\Delta$, $\lambda$, $V(r)$ etc.) in units of the hopping parameter~$t$ and spatial variables in units of the lattice constant~$a$.

%-----------------------------------------------------------------------------------
%-----------------------------------------------------------------------------------
\subsection{$A$ exciton in $K$ valley}
%-----------------------------------------------------------------------------------
%-----------------------------------------------------------------------------------

Consider a special case $s_e=\tau_e=1$, $s_h=\tau_h=-1$ which correspond to the $A$ exciton in $K$ valley. Then,

\setlength\arraycolsep{7pt}
\renewcommand*{\arraystretch}{2}
\begin{equation}
\hat{H} =
\begin{pmatrix}
-V(r) &  e^{-i \varphi}D^-_h & -e^{-i \varphi}D^+_e & 0 \\
e^{i\varphi}D^+_h & \Delta_h-V(r) & 0 & -e^{-i\varphi}D^+_e \\
-e^{i\varphi}D^-_e & 0 & -\Delta_e-V(r) & e^{-i\varphi}D^-_h \\
0 & -e^{i\varphi}D^-_e & e^{i\varphi}D^+_h & -V(r)
\end{pmatrix},
\label{H}
\end{equation}

Using the ansatz
$\psi(r,\varphi) = e^{i l \varphi} \left(  \psi^l_1(r) e^{-i\varphi}, \,\psi^l_2(r), \,\psi^l_3(r), \,\psi^l_4(r) e^{i\varphi}\right)^T $, the Hamiltonian~\eqref{H} acting on \{$\psi^l_1(r)$, $\psi^l_2(r)$, $\psi^l_3(r)$, $\psi^l_4(r)$\} takes the form
\setlength\arraycolsep{7pt}
\renewcommand*{\arraystretch}{2}
\begin{equation}
\hat{H}_l =
\begin{pmatrix}
-V(r) &  -i\p_r - \frac{il}{r} & -i\p_r - \frac{il}{r} & 0 \\
(-i\p_r)^\dagger + \frac{il}{r} & \Delta_h-V(r) & 0 & (-i\p_r)^\dagger - \frac{il}{r} \\
(-i\p_r)^\dagger + \frac{il}{r} & 0 & -\Delta_e-V(r) & (-i\p_r)^\dagger - \frac{il}{r} \\
0 & -i\p_r + \frac{il}{r} & -i\p_r + \frac{il}{r} & -V(r)
\end{pmatrix},
\label{Hl}
\end{equation}
where $(-i\p_r)^\dagger = -i\p_r-i/r$, or, equivalently, $\p_r^\dagger = -\p_r-1/r$. Note that the Hamiltonian~\eqref{Hl} is self-adjoint with respect to the scalar product
$$
\langle \pmb\psi, \pmb\chi \rangle = \sum_{i=1}^4 \int_0^\infty \psi^*_i(r) \chi_i(r)\, r dr.
$$
Let us first consider the $l=0$ state for which
\setlength\arraycolsep{7pt}
\renewcommand*{\arraystretch}{2}
\begin{equation}
\hat{H}_0 =
\begin{pmatrix}
-V(r) &  -i\p_r & -i\p_r & 0 \\
(-i\p_r)^\dagger & \Delta_h-V(r) & 0 & (-i\p_r)^\dagger \\
(-i\p_r)^\dagger  & 0 & -\Delta_e-V(r) & (-i\p_r)^\dagger \\
0 & -i\p_r & -i\p_r & -V(r)
\end{pmatrix},
\label{Hl}
\end{equation}
Introduce the Hankel transforms (for simplicity we will omit the superscript $l=0$),
$$
\psi_{1,4}(r)=\int_0^\infty F_{1,4}(k) J_1(kr) k dk
\quad\text{and}\quad
\psi_{2,3}(r)=\int_0^\infty F_{2,3}(k) J_0(kr) k dk,
$$
%Note, that the Hankel transform satisfy
where the inverse transforms are given by
$$
F_{1,4}(k)=\int_0^\infty \psi_{1,4}(r) J_1(kr) r dr
\quad\text{and}\quad
F_{2,3}(k)=\int_0^\infty \psi_{2,3}(r) J_0(kr) r dr.
$$
The transformed Hamiltonian acting on $\mathbf{F}\equiv$ \{$F_1(k)$, $F_2(k)$, $F_3(k)$, $F_4(k)$\} is
\setlength\arraycolsep{7pt}
\renewcommand*{\arraystretch}{2}
\begin{equation}
\hat{H}_0 =
\begin{pmatrix}
-\int_0^\infty \tilde{V}_0(k,k') \;...\; k'dk' &  ik & ik & 0 \\
-ik & \Delta_h-\int_0^\infty \tilde{V}_1(k,k') \;...\; k'dk' & 0 & -ik \\
-ik  & 0 & -\Delta_e-\int_0^\infty \tilde{V}_1(k,k') \;...\; k'dk' & -ik \\
0 & ik & ik & -\int_0^\infty \tilde{V}_0(k,k') \;...\; k'dk'
\end{pmatrix},
\label{Hl}
\end{equation}
where
$$
\tilde{V}_{0,1}(k,k') \equiv \int_0^\infty V(r) J_{0,1}(kr) J_{0,1}(k'r) r dr.
$$
%which is manifestly symmetric $\tilde{V}_{0,1}(k,k')=\tilde{V}_{0,1}(k',k)$, and it is known that
%$$\tilde{V}_0(0,k')k'=\frac{1}{1+r_0 k'}$$
%where $r_0$ is the screening length. We can then write
%$$\tilde{V}_{0}(k,k') k' = \frac{1}{1+r_0 k'} + k' \int_0^\infty V(r) \left[J_{0}(kr)-1\right] J_{0}(k'r) r dr.$$

%Note, that as the expected exciton states are bounded,
%$$\int_0^\infty \psi_i(r)rdr<\infty,$$
%we have $|F_i(0)|<\infty$. Thus, $F_i(0)$ are unimportant as they do not influence $\psi_i(r)$ and we can only consider $k>0$.

The matrix~\eqref{Hl} acting in space $\mathbb{R}_{>0}\times \mathbb{R}_{>0}\times \mathbb{R}_{>0} \times \mathbb{R}_{>0}$ is not Hermitian because of the integrals $\int_0^\infty \tilde{V}_{0,1}(k,k') F_i(k')\, k'dk'$. Nevertheless, the Hamiltonian can be written as a product
$
\hat{H}_0=\hat{A}\hat{K}
$
with Hermitian $\hat{A}$ and a positive-definite diagonal matrix $\hat{K}$ containing the $k$-numbers on the diagonal. 
The resulting generalized eigenvalue problem
$$
\hat{A}\hat{K} \mathbf{F}_n = \lambda_n \mathbf{F}_n
$$
ensures that 
%$\lambda_n \in \mathbb{R}$\footnote
the eigenvalues $\lambda_n$ are real~\footnote{One can convert the system to the conventional eigenvalue problem 
$\hat{K}^{1/2}\hat{A}\hat{K}^{1/2} \tilde{\mathbf{F}}_n = \lambda_n \tilde{\mathbf{F}}_n$ with $\tilde{\mathbf{F}}_n\equiv \hat{K}^{1/2}\mathbf{F}_n$ and Hermitian matrix $\hat{K}^{1/2}\hat{A}\hat{K}^{1/2}$. However, we do not use this as the numerical tools for solving the generalized eigenvalue systems are provided in Python.}. The Hermitian matrix $A$ acting on  $\hat{K} \mathbf{F}_n$ is
\setlength\arraycolsep{7pt}
\renewcommand*{\arraystretch}{2}
\begin{equation}
\hat{A} =
\begin{pmatrix}
-\int_0^\infty \tilde{V}_0(k,k') \;...\; dk' &  i & i & 0 \\
-i & \frac{\Delta_h}{k}-\int_0^\infty \tilde{V}_1(k,k') \;...\; dk' & 0 & -i \\
-i  & 0 & -\frac{\Delta_e}{k}-\int_0^\infty \tilde{V}_1(k,k') \;...\; dk' & -i \\
0 & i & i & -\int_0^\infty \tilde{V}_0(k,k') \;...\; dk'
\end{pmatrix},
\label{Amatrix}
\end{equation}

%\setlength\arraycolsep{2pt}
%\renewcommand*{\arraystretch}{2}
%\begin{multline}
%\hat{K}^{1/2}\hat{A}\hat{K}^{1/2} = 
%\\
%\begin{pmatrix}
%-\int_0^\infty \tilde{V}_0(k,k') \;...\; \sqrt{kk'}\, dk' &  ik & ik & 0 \\
%-ik & \Delta_h-\int_0^\infty \tilde{V}_1(k,k') \;...\; \sqrt{kk'}\,dk' & 0 & -ik \\
%-ik  & 0 & -\Delta_e-\int_0^\infty \tilde{V}_1(k,k') \;...\; \sqrt{kk'}\, dk' & -ik \\
%0 & ik & ik & -\int_0^\infty \tilde{V}_0(k,k') \;...\; \sqrt{kk'}\,dk'
%\end{pmatrix},
%\label{Hl}
%\end{multline}

\appendix

%-----------------------------------------------------------------------------------
%-----------------------------------------------------------------------------------
\section{Evaluation of the potentials}
%-----------------------------------------------------------------------------------
%-----------------------------------------------------------------------------------


We need to evaluate the following integrals:
\medskip

{\bf Integral 1}:
$$
\tilde{V}_{0}(k,k') \equiv \int_0^\infty V(r) J_{0}(kr) J_{0}(k'r) r dr.
$$
Using~\eqref{Keldysh}, integral 1 we can write in the form
$$
\tilde{V}_{0}(k,k') \equiv \int_0^\infty dk'' k'' \tilde{V}(k'') \int_0^\infty J_{0}(kr) J_{0}(k'r) J_0(k'' r) r dr.
$$
Using the formula from the book~\cite{Watson-book} and the orthogonality of the Bessel functions the integral is reduced to
\begin{equation}
\tilde{V}_{0}(k,k') = \frac{1}{\pi} \int_0^\pi  \tilde{V}\left(q\right)\, d\varphi,
\quad\text{where}\quad
q\equiv |\mathbf{k}-\mathbf{k}'|=\sqrt{k^2+k'^2-2kk'\cos\varphi}.
\label{V0kk'}
\end{equation}
Using the explicit form of $\tilde{V}(q)$,
\begin{equation*}
\boxed{
\tilde{V}_{0}(k,k') = \frac{\mathcal{V}_0}{\pi} \int_0^\pi  \frac{1}{q(1+r_0 q)}\, d\varphi.
}
\end{equation*}
It is easy to see that when either $k$ or $k'$ are equal to zero, we restore the familiar Keldysh potential in the momentum space, $\tilde{V}_{0}(k,0)= \tilde{V}_{0}(0,k) = \tilde{V}(k)$. Useful note: compare this to the similar integral $I_{k,k'}$ in the Eq.(B8) in Ref.~\cite{Wu-PRB-2015} which arises when solving the Bethe-Salpeter equation.

It may appear as though the integral~\eqref{V0kk'} is divergent due to the logarithmic singularity at $k'=k$. However, note that~\eqref{V0kk'} enters the integration
$$
\int_0^\infty \tilde{V}_{0}(k,k') F(k') k' dk' = 
\frac{1}{2\pi} \int_0^\infty  \int_0^{2\pi} \tilde{V}\left(|\mathbf{k}-\mathbf{k}'|\right) F(k')d\varphi k' dk'
= \frac{1}{2\pi} \iint \tilde{V}(q) F(|\mathbf{k}-\mathbf{q}|) d\mathbf{q}.
$$
%This can be transformed to the Cartesian coordinates.
It is easy to see that the last integral is convergent as $d\mathbf{q}=q\,dq\,d\varphi$ cancels the Coulomb $1/q$ singularity of $\tilde{V}(q)$. 

\medskip

{\bf Integral 2}:
$$
\tilde{V}_{1}(k,k') \equiv \int_0^\infty V(r) J_{1}(kr) J_{1}(k'r) r dr.
$$
Again, using~\eqref{Keldysh}, integral 2 we can write in the form
$$
\tilde{V}_{1}(k,k') \equiv \int_0^\infty dk'' k'' \tilde{V}(k'') \int_0^\infty J_{1}(kr) J_{1}(k'r) J_0(k'' r) r dr.
$$
Using the same formula from the book~\cite{Watson-book}, 
%and the orthogonality of the Bessel functions 
the integral is transformed to
\begin{equation}
\tilde{V}_{1}(k,k') = \frac{kk'}{\pi} \int_0^\infty dk'' k'' \tilde{V}(k'')  \int_0^\infty \int_0^\pi   
\frac{J_{1}\left(q r\right) J_0(k'' r)}{q} \,r^2\sin^2\varphi\; d\varphi\,  dr,
\quad\text{where}\quad q=|\mathbf{k}-\mathbf{k}'|
\label{V1kk'}
\end{equation}
Using the property $J_1(x)=-J'_0(x)$ and the orthogonality of Bessel functions,
\begin{multline}
\tilde{V}_{1}(k,k') = -\, \frac{kk'}{\pi}     \int_0^\pi d\varphi\,\sin^2\varphi\;
\frac{1}{q} \frac{d}{dq} \int_0^\infty dk'' k''\tilde{V}(k'') \int_0^\infty dr\,r J_0\left(q r\right) J_0(k'' r) \\
= -\, \frac{kk'}{\pi}     \int_0^\pi d\varphi\,\sin^2\varphi\;
\frac{1}{q} \frac{d}{dq} \int_0^\infty dk'' k''\tilde{V}(k'') \frac{\delta(q-k'')}{q} \\
= -\, \frac{kk'}{\pi}     \int_0^\pi d\varphi\,\sin^2\varphi\;
\frac{\tilde{V}'(q)}{q}
\end{multline}
Finally, using the expression for $\tilde{V}(q)$,
$$
\boxed{
\tilde{V}_{1}(k,k') = \mathcal{V}_0\,\frac{kk'}{\pi}     \int_0^\pi 
\frac{1+2r_0q}{q^3(1+r_0q)^2}\, \sin^2\varphi\; d\varphi
}
$$
As before, it is possible to show that the double integral over $k' dk' d\varphi$ converges. At $q\to 0$ in a small circle around $\mathbf{k}$ we have
$$
\int \frac{|\mathbf{k}\times\mathbf{k}'|^2}{|\mathbf{k}-\mathbf{k}'|^3}\,d^2\mathbf{k}'
=\int \frac{|\mathbf{k}\times\mathbf{q}|^2}{q^3}\,d^2\mathbf{q} < \infty
$$

\subsection{Numerical evaluation of the eigensystem}


For numerical evaluation we take a uniform mesh $k_n=\Delta k\, n$ for $1\le n\le M$. 
%Using the uniform mesh $k_n=\Delta k\, n$ for $0\le n\le M$, 
The integrals are discretized using the quadrature (for simplicity we drop the indices here)
$$
\int_0^\infty \tilde{V}(k_n,k')\, F(k')\, k'\, dk'
\approx
\sum_{m=1}^{M} w_{nm} F(k_m)k_m
$$
where
$$
w_{nm}= \tilde{V}(k_n,k_m) \Delta k \quad\text{for}\quad n\neq m,
$$
and
$$
w_{nn}=\frac{1}{k_n}\int_{k_n-\frac12\Delta k}^{k_n+\frac12\Delta k}
\tilde{V}^{\rm (sing)}(k_n,k')\, k' \,dk' 
+ \tilde{V}^{\rm (nonsing)}(k_n,k_n) \, \Delta k
$$
otherwise, where we have splitted the potential into a singular $ \tilde{V}^{\rm (sing)}(k_n,k')$ and nonsingular  $\tilde{V}^{\rm (nonsing)}(k_n,k')$ parts to deal with the Coloumb singularity. Potentials $\tilde{V}_0(k,k')$ and $\tilde{V}_1(k,k')$ are splitted into singular and nonsingular parts by partial fractions as follows
\begin{equation*}
\tilde{V}_{0}(k,k') = \frac{\mathcal{V}_0}{\pi} \int_0^\pi \frac{1}{q}\,d\varphi
-
\frac{\mathcal{V}_0}{\pi} \int_0^\pi  \frac{r_0}{1+r_0q} \, d\varphi
\end{equation*}
and
\begin{equation*}
\tilde{V}_{1}(k,k') = \mathcal{V}_0\,\frac{kk'}{\pi} \int_0^\pi \left( \frac{1}{q^3} - \frac{r_0^2}{q}\right)\sin^2\varphi\,d\varphi
+
\mathcal{V}_0\,\frac{kk'}{\pi} \int_0^\pi  \frac{r_0^3 \left(2+r_0q\right)}{\left(1+r_0q\right)^2} \sin^2\varphi\, d\varphi.
\end{equation*}
Integrals of the singular parts are simple enough to be evaluated exactly in terms of the elliptic functions.

%--------------------------------------------------------------------------------------------
%--------------------------------------------------------------------------------------------
%\seciton{Acknowledgments}
%--------------------------------------------------------------------------------------------
%--------------------------------------------------------------------------------------------
%The work is supported by the Russian Science Foundation under the grant 18-12-00429.

\bibliography{bibliography-excitons}


%%%%%%%%%
%\fi
%%%%%%%%%%%%%5

\end{document}


\iffigures
\begin{figure}[t!]
	\begin{center}
\includegraphics[width=3.4in]{fig1.png}
		\caption{\label{fig:tffo} 
Figure caption.
		}		
	\end{center}		
\end{figure}
\fi
